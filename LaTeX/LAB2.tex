%% Необхоимый минимум
\documentclass[a4paper,14pt]{article}



%%% Работа с русским языком
\usepackage{cmap}					% поиск в PDF
\usepackage{mathtext} 				% русские буквы в формулах
\usepackage[T2A]{fontenc}			% кодировка
\usepackage[utf8]{inputenc}			% кодировка исходного текста
\usepackage[english,russian]{babel}	% локализация и переносы



%%% Дополнительная работа с математикой
\usepackage{amsfonts,amssymb,amsthm,mathtools} % AMS
\usepackage{amsmath}
\usepackage{icomma} % "Умная" запятая: $0,2$ --- число, $0, 2$ --- перечисление

%% Номера формул
\mathtoolsset{showonlyrefs=true} % Показывать номера только у тех формул, на которые есть \eqref{} в тексте.

%% Шрифты
\usepackage{euscript}	 % Шрифт Евклид
\usepackage{mathrsfs} % Красивый матшрифт

%% Свои команды
\DeclareMathOperator{\sgn}{\mathop{sgn}}

%% Перенос знаков в формулах (по Львовскому)
\newcommand*{\hm}[1]{#1\nobreak\discretionary{}
	{\hbox{$\mathsurround=0pt #1$}}{}}

\usepackage{gensymb}
\usepackage{indentfirst}
\usepackage{lscape}

\usepackage{caption}
\captionsetup{labelsep=period}
\captionsetup{justification=centerlast}
\usepackage{subcaption}
\renewcommand{\thesubfigure}{\asbuk{subfigure}}

\usepackage{microtype}


%%% Параметры страницы: поля, колонтитулы, интерлиньяж
\usepackage{extsizes} % Возможность сделать 14-й шрифт
\usepackage{geometry} % Простой способ задавать поля
\geometry{top=20mm}
\geometry{bottom=20mm}
\geometry{left=30mm}
\geometry{right=15mm}

%\usepackage{fancyhdr} % Колонтитулы
%\pagestyle{fancy}
%\renewcommand{\headrulewidth}{0mm}  % Толщина линейки, отчеркивающей верхний колонтитул
%\lfoot{Нижний левый}
%\rfoot{Нижний правый}
%\rhead{Верхний правый}
%\chead{Верхний в центре}
%\lhead{Верхний левый}
% \cfoot{Нижний в центре} % По умолчанию здесь номер страницы

\usepackage{setspace} % Интерлиньяж
\onehalfspacing % Интерлиньяж 1.5
%\doublespacing % Интерлиньяж 2
%\singlespacing % Интерлиньяж 1

%%% Работа с таблицами
\usepackage{array,tabularx,tabulary,booktabs} % Дополнительная работа с таблицами
\usepackage{longtable}  % Длинные таблицы
\usepackage{multirow} % Слияние строк в таблице

%%% Работа с картинками
\usepackage{graphicx}  % Для вставки рисунков
\graphicspath{{images/}{images2/}}  % папки с картинками
\setlength\fboxsep{3pt} % Отступ рамки \fbox{} от рисунка
\setlength\fboxrule{1pt} % Толщина линий рамки \fbox{}
\usepackage{wrapfig} % Обтекание рисунков и таблиц текстом

\usepackage{lastpage} % Узнать, сколько всего страниц в документе.

\usepackage{soul} % Модификаторы начертания
\usepackage{soulutf8} % Модификаторы начертания

%\renewcommand{\familydefault}{\sfdefault} % Начертание шрифта

\usepackage{multicol} % Несколько колонок
\usepackage{makecell}
\newlength{\bibitemsep}\setlength{\bibitemsep}{.2\baselineskip plus .05\baselineskip minus .05\baselineskip}
\newlength{\bibparskip}\setlength{\bibparskip}{0pt}
\let\oldthebibliography\thebibliography
\renewcommand\thebibliography[1]{%
	\oldthebibliography{#1}%
	\setlength{\parskip}{\bibitemsep}%
	\setlength{\itemsep}{\bibparskip}%
}

\newcommand{\rectangle}[2]{{% #1 = width, #2 = height
		\fboxsep=-\fboxrule\sbox0{}\wd0=#1\ht0=#2\relax\fbox{\box0}}}

\usepackage{tikz}
\usepackage{multirow}
\usepackage{graphicx}
\usepackage{enumitem}
\usepackage{listings}

\begin{document}
	
\newcommand{\signfield}[2]{%
	\rule{#1}{0.1pt}\hspace{-#1}
	\raisebox{-1.1ex}{\parbox[t]{#1}{\centering\scriptsize{#2}}}
}

\thispagestyle{empty}
\begin{center}
	\large
	{\scshape министерство науки и высшего образования \\Российской Федерации}
	\vspace{1ex}
	
	{\scshape национальный исследовательский ядерный\\университет <<мифи>>}
	\vspace{2.5 cm}
	
	\large
	{\scshape \textbf{О\,Т\,Ч\,Е\,Т}\\о выполнении курсового проекта}
	{\scshape \textbf{<<теплогидравлический и нейтронно-физический расчет активной зоны реактора ВВЭР-1200>>}}
\end{center}

\vspace{2cm}
\begin{flushright}
	\textbf{Выполнил}\\студент группы С19-103\\
	Мамлеев А.\,А.\\[1ex]
	\signfield{3cm}{подпись}
\end{flushright}


\vspace{1 ex}
\parbox{0.5\linewidth}{
\begin{flushleft}
	\textbf{Научный руководитель\\по теплогидравлическому\\расчету}\\канд. техн. наук Маслов Ю.\,А.\\[1ex]
	\signfield{3cm}{подпись}
\end{flushleft}
}
\parbox{0.5\linewidth}{
\begin{flushleft}
	\textbf{Научный руководитель\\по нейтронно-физическому\\расчету}\\канд. физ.-мат. наук Савандер В.\,И.\\[1ex]
	\signfield{3cm}{подпись}
\end{flushleft}
}

\vspace{1 ex}
\textbf{Оценка (ECTS): }



\vfill

\begin{center}
	Москва 2024
\end{center}
\newpage


\textbf{Цель:} изучить методы синтеза комбинационных схем на логических элементах; получить навыки проектирования комбинационных схем на VHDL; овладеть инструментальными средствами проектирования схем на ПЛИС; приобрести опыт экспериментального исследования синтезируемых схем.

\section{Синтез комбинационной схемы}

В соответствии с вариантом дана следующая система ФАЛ:
\begin{equation} \label{eq: FAL}
	\left\{
	\begin{aligned}
		F_1(x_3, x_2, x_1, x_0) &= \sum(1, 3, 5, 7, 8, 10, 12, 14),\\
		F_2(x_3, x_2, x_1, x_0) &= \sum(1, 2, 3, 5, 7, 9, 15),\\
		F_3(x_3, x_2, x_1, x_0) &= \sum(1, 3, 5, 7, 8, 9).
	\end{aligned}
	\right.
\end{equation}

Для данных функций составим таблицу истинности (табл.~\ref{tab: true-false}).
\begin{table}[h]
	\centering
	\caption{Таблица истинности для системы ФАЛ \eqref{eq: FAL}} \label{tab: true-false}
	\begin{tabular}{|c||c|c|c|c||c|c|c|}
		\hline
		№ & $x_3$ & $x_2$ & $x_1$ & $x_0$ & $F_1$ & $F_2$ & $F_3$ \\
		\hline
		0 & 0 & 0 & 0 & 0 & 0 & 0 & 0 \\
		\hline
		1 & 0 & 0 & 0 & 1 & 1 & 1 & 1 \\
		\hline
		2 & 0 & 0 & 1 & 0 & 0 & 1 & 1 \\
		\hline
		3 & 0 & 0 & 1 & 1 & 1 & 1 & 1 \\
		\hline
		4 & 0 & 1 & 0 & 0 & 0 & 0 & 0 \\
		\hline
		5 & 0 & 1 & 0 & 1 & 1 & 1 & 1 \\
		\hline
		6 & 0 & 1 & 1 & 0 & 0 & 0 & 0 \\
		\hline
		7 & 0 & 1 & 1 & 1 & 1 & 1 & 1 \\
		\hline
		8 & 1 & 0 & 0 & 0 & 1 & 0 & 1 \\
		\hline
		9 & 1 & 0 & 0 & 1 & 0 & 1 & 1 \\
		\hline
		10 & 1 & 0 & 1 & 0 & 1 & 0 & 0 \\
		\hline
		11 & 1 & 0 & 1 & 1 & 0 & 0 & 0 \\
		\hline
		12 & 1 & 1 & 0 & 0 & 1 & 0 & 0 \\
		\hline
		13 & 1 & 1 & 0 & 1 & 0 & 0 & 0 \\
		\hline
		14 & 1 & 1 & 1 & 0 & 1 & 0 & 0 \\
		\hline
		15 & 1 & 1 & 1 & 1 & 0 & 1 & 0 \\
		\hline
	\end{tabular}
\end{table}

\subsection{Минимизация}
Произведем минимизацию функций методом диаграмм Вейча; на рис.~\ref{fig: diagram_sample} представлена эталонная диаграмма, которой далее будем пользоваться.
\begin{figure}[h]
	\centering
\begin{tikzpicture}
	\draw (0,0) grid (4,4);
	\node at (3.5, .5){0};
	\node at (2.5,.5){1};
	\node at (3.5, 1.5){2};
	\node at (2.5, 1.5){3};
	\node at (0.5, 0.5){4};
	\node at (1.5, .5){5};
	\node at (0.5, 1.5){6};
	\node at (1.5,1.5){7};
	\node at (3.5,3.5){8};
	\node at (2.5, 3.5){9};
	\node at (3.5, 2.5){10};
	\node at (2.5, 2.5){11};
	\node at (0.5, 3.5) {12};
	\node at (1.5,3.5){13};
	\node at (0.5, 2.5){14};
	\node at (1.5, 2.5){15};
	\draw (0,4.3) --node[midway, above]{$x_2$} (2,4.3);
	\draw (-0.3,4) --node[midway, left]{$x_3$} (-0.3,2);
	\draw (1,-0.3) --node[midway, below]{$x_0$} (3,-0.3);
	\draw (4.3,1) --node[midway, right]{$x_1$} (4.3,3);
\end{tikzpicture}
\caption{Эталонная диаграмма Вейча} \label{fig: diagram_sample}
\end{figure}

\begin{figure}[h]
	\captionsetup[subfigure]{justification=centerlast}
	\begin{subfigure}{0.3\linewidth}
		\begin{tikzpicture}[scale=0.8]
			\draw (0,0) grid (4,4);
			%\node at (3.5, .5){0};
			\node at (2.5,.5){1};
			%\node at (3.5, 1.5){2};
			\node at (2.5, 1.5){1};
			%\node at (0.5, 0.5){4};
			\node at (1.5, .5){1};
			%\node at (0.5, 1.5){6};
			\node at (1.5,1.5){1};
			\node at (3.5,3.5){1};
			%\node at (2.5, 3.5){9};
			\node at (3.5, 2.5){1};
			%\node at (2.5, 2.5){11};
			\node at (0.5, 3.5) {1};
			%\node at (1.5,3.5){13};
			\node at (0.5, 2.5){1};
			%\node at (1.5, 2.5){15};
			\draw (0,4.3) --node[midway, above]{$x_2$} (2,4.3);
			\draw (-0.3,4) --node[midway, left]{$x_3$} (-0.3,2);
			\draw (1,-0.3) --node[midway, below]{$x_0$} (3,-0.3);
			\draw (4.3,1) --node[midway, right]{$x_1$} (4.3,3);
		\end{tikzpicture}
		\caption{Для функции $F_1$} \label{fig:Veitch_F1}
	\end{subfigure}
	\hfill
	\begin{subfigure}{0.3\linewidth}
		\centering
		\begin{tikzpicture}[scale=0.8]
			\draw (0,0) grid (4,4);
			%\node at (3.5, .5){0};
			\node at (2.5,.5){1};
			\node at (3.5, 1.5){1};
			\node at (2.5, 1.5){1};
			%\node at (0.5, 0.5){4};
			\node at (1.5, .5){1};
			%\node at (0.5, 1.5){6};
			\node at (1.5,1.5){1};
			%\node at (3.5,3.5){8};
			\node at (2.5, 3.5){1};
			%\node at (3.5, 2.5){10};
			%\node at (2.5, 2.5){11};
			%\node at (0.5, 3.5) {12};
			%\node at (1.5,3.5){13};
			%\node at (0.5, 2.5){14};
			\node at (1.5, 2.5){1};
			\draw (0,4.3) --node[midway, above]{$x_2$} (2,4.3);
			\draw (-0.3,4) --node[midway, left]{$x_3$} (-0.3,2);
			\draw (1,-0.3) --node[midway, below]{$x_0$} (3,-0.3);
			\draw (4.3,1) --node[midway, right]{$x_1$} (4.3,3);
		\end{tikzpicture}
		\caption{Для функции $F_2$}\label{fig:Veitch_F2}
	\end{subfigure}
	\hfill
	\begin{subfigure}{0.3\linewidth}
		\centering
		\begin{tikzpicture}[scale=0.8]
			\draw (0,0) grid (4,4);
			%\node at (3.5, .5){0};
			\node at (2.5,.5){1};
			%\node at (3.5, 1.5){2};
			\node at (2.5, 1.5){1};
			%\node at (0.5, 0.5){4};
			\node at (1.5, .5){1};
			%\node at (0.5, 1.5){6};
			\node at (1.5,1.5){1};
			\node at (3.5,3.5){1};
			\node at (2.5, 3.5){1};
			%\node at (3.5, 2.5){10};
			%\node at (2.5, 2.5){11};
			%\node at (0.5, 3.5) {12};
			%\node at (1.5,3.5){13};
			%\node at (0.5, 2.5){14};
			%\node at (1.5, 2.5){15};
			\draw (0,4.3) --node[midway, above]{$x_2$} (2,4.3);
			\draw (-0.3,4) --node[midway, left]{$x_3$} (-0.3,2);
			\draw (1,-0.3) --node[midway, below]{$x_0$} (3,-0.3);
			\draw (4.3,1) --node[midway, right]{$x_1$} (4.3,3);
		\end{tikzpicture}
		\caption{Для функции $F_3$}\label{fig:Veitch_F3}
	\end{subfigure}
	\caption{Диаграммы Вейча для заданных функций}\label{fig:Veitch_all}
\end{figure}

Произведя минимизацию при помощи диаграмм Вейча (рис.~\ref{fig:Veitch_all}), запишем заданные функции в форме МДНФ:
\begin{equation} \label{eq:sys}
	\left\{
	\begin{aligned} 
		F_1(x_3, x_2, x_1, x_0) &= x_0\bar x_3 \vee \bar x_0 x_3,\\
		F_2(x_3, x_2, x_1, x_0) &= x_0\bar x_3 \vee x_0 x_1 x_2 \vee x_0\bar x_1 \bar x_2 \vee x_1\bar x_2\bar x_3,\\
		F_3(x_3, x_2, x_1, x_0) &= x_0\bar x_3 \vee \bar x_1\bar x_2 x_3.
	\end{aligned}
	\right.
\end{equation}

\begin{landscape}
	\begin{table}[]
		\caption{Импликантная матрица системы логических функций} \label{tab:matrix}
		\centering
		\footnotesize
		\begin{tabular}{|c|ccccccccccccccccccccc|}
			\hline
			\multirow{4}{*}{\makecell{Импли-\\канта}} &
			\multicolumn{21}{c|}{Конституента} \\ \cline{2-22} 
			&
			\multicolumn{3}{c|}{1} &
			\multicolumn{1}{c|}{2} &
			\multicolumn{3}{c|}{3} &
			\multicolumn{3}{c|}{5} &
			\multicolumn{3}{c|}{7} &
			\multicolumn{2}{c|}{8} &
			\multicolumn{2}{c|}{9} &
			\multicolumn{1}{c|}{10} &
			\multicolumn{1}{c|}{12} &
			\multicolumn{1}{c|}{14} &
			15 \\ \cline{2-22} &
			\multicolumn{3}{c|}{$\bar x_3 \bar x_2 \bar x_1 x_0$} &
			\multicolumn{1}{c|}{$\bar x_3 \bar x_2 \bar x_1 \bar x_0$} &
			\multicolumn{3}{c|}{$\bar x_3 \bar x_2 x_1 x_0$} &
			\multicolumn{3}{c|}{$\bar x_3 x_2 \bar x_1 x_0$} &
			\multicolumn{3}{c|}{$\bar x_3 x_2 x_1 x_0$} &
			\multicolumn{2}{c|}{$x_3 \bar x_2 \bar x_1 \bar x_0$} &
			\multicolumn{2}{c|}{$x_3 \bar x_2 \bar x_1 x_0$} &
			\multicolumn{1}{c|}{$x_3 \bar x_2 x_1 \bar x_0$} &
			\multicolumn{1}{c|}{$x_3 x_2 \bar x_1 \bar x_0$} &
			\multicolumn{1}{c|}{$x_3 x_2 x_1 \bar x_0$} &
			$x_3 x_2 x_1 x_0$ \\ \cline{2-22} 
			&
			\multicolumn{1}{c|}{$F_1$} &
			\multicolumn{1}{c|}{$F_2$} &
			\multicolumn{1}{c|}{$F_3$} &
			\multicolumn{1}{c|}{$F_2$} &
			\multicolumn{1}{c|}{$F_1$} &
			\multicolumn{1}{c|}{$F_2$} &
			\multicolumn{1}{c|}{$F_3$} &
			\multicolumn{1}{c|}{$F_1$} &
			\multicolumn{1}{c|}{$F_2$} &
			\multicolumn{1}{c|}{$F_3$} &
			\multicolumn{1}{c|}{$F_1$} &
			\multicolumn{1}{c|}{$F_2$} &
			\multicolumn{1}{c|}{$F_3$} &
			\multicolumn{1}{c|}{$F_1$} &
			\multicolumn{1}{c|}{$F_3$} &
			\multicolumn{1}{c|}{$F_2$} &
			\multicolumn{1}{c|}{$F_3$} &
			\multicolumn{1}{c|}{$F_1$} &
			\multicolumn{1}{c|}{$F_1$} &
			\multicolumn{1}{c|}{$F_1$} &
			$F_2$ \\ \hline
			\makecell{$\bar x_3 x_0$\\$F_1 F_2 F_3$} &
			\multicolumn{1}{c|}{} &
			\multicolumn{1}{c|}{} &
			\multicolumn{1}{c|}{} &
			\multicolumn{1}{c|}{} &
			\multicolumn{1}{c|}{} &
			\multicolumn{1}{c|}{} &
			\multicolumn{1}{c|}{} &
			\multicolumn{1}{c|}{} &
			\multicolumn{1}{c|}{} &
			\multicolumn{1}{c|}{} &
			\multicolumn{1}{c|}{} &
			\multicolumn{1}{c|}{} &
			\multicolumn{1}{c|}{} &
			\multicolumn{1}{c|}{} &
			\multicolumn{1}{c|}{} &
			\multicolumn{1}{c|}{} &
			\multicolumn{1}{c|}{} &
			\multicolumn{1}{c|}{} &
			\multicolumn{1}{c|}{} &
			\multicolumn{1}{c|}{} &
			\\ \hline
			\makecell{$x_3 \bar x_0$\\$F_1$} &
			\multicolumn{1}{c|}{} &
			\multicolumn{1}{c|}{} &
			\multicolumn{1}{c|}{} &
			\multicolumn{1}{c|}{} &
			\multicolumn{1}{c|}{} &
			\multicolumn{1}{c|}{} &
			\multicolumn{1}{c|}{} &
			\multicolumn{1}{c|}{} &
			\multicolumn{1}{c|}{} &
			\multicolumn{1}{c|}{} &
			\multicolumn{1}{c|}{} &
			\multicolumn{1}{c|}{} &
			\multicolumn{1}{c|}{} &
			\multicolumn{1}{c|}{} &
			\multicolumn{1}{c|}{} &
			\multicolumn{1}{c|}{} &
			\multicolumn{1}{c|}{} &
			\multicolumn{1}{c|}{} &
			\multicolumn{1}{c|}{} &
			\multicolumn{1}{c|}{} &
			\\ \hline
			\makecell{$x_2 x_1 x_0$\\$F_2$} &
			\multicolumn{1}{c|}{} &
			\multicolumn{1}{c|}{} &
			\multicolumn{1}{c|}{} &
			\multicolumn{1}{c|}{} &
			\multicolumn{1}{c|}{} &
			\multicolumn{1}{c|}{} &
			\multicolumn{1}{c|}{} &
			\multicolumn{1}{c|}{} &
			\multicolumn{1}{c|}{} &
			\multicolumn{1}{c|}{} &
			\multicolumn{1}{c|}{} &
			\multicolumn{1}{c|}{} &
			\multicolumn{1}{c|}{} &
			\multicolumn{1}{c|}{} &
			\multicolumn{1}{c|}{} &
			\multicolumn{1}{c|}{} &
			\multicolumn{1}{c|}{} &
			\multicolumn{1}{c|}{} &
			\multicolumn{1}{c|}{} &
			\multicolumn{1}{c|}{} &
			\\ \hline
			\makecell{$\bar x_2 \bar x_1 x_0$\\$F_2$} &
			\multicolumn{1}{c|}{} &
			\multicolumn{1}{c|}{} &
			\multicolumn{1}{c|}{} &
			\multicolumn{1}{c|}{} &
			\multicolumn{1}{c|}{} &
			\multicolumn{1}{c|}{} &
			\multicolumn{1}{c|}{} &
			\multicolumn{1}{c|}{} &
			\multicolumn{1}{c|}{} &
			\multicolumn{1}{c|}{} &
			\multicolumn{1}{c|}{} &
			\multicolumn{1}{c|}{} &
			\multicolumn{1}{c|}{} &
			\multicolumn{1}{c|}{} &
			\multicolumn{1}{c|}{} &
			\multicolumn{1}{c|}{} &
			\multicolumn{1}{c|}{} &
			\multicolumn{1}{c|}{} &
			\multicolumn{1}{c|}{} &
			\multicolumn{1}{c|}{} &
			\\ \hline
			\makecell{$\bar x_3 \bar x_2 x_1$\\$F_2$} &
			\multicolumn{1}{c|}{} &
			\multicolumn{1}{c|}{} &
			\multicolumn{1}{c|}{} &
			\multicolumn{1}{c|}{} &
			\multicolumn{1}{c|}{} &
			\multicolumn{1}{c|}{} &
			\multicolumn{1}{c|}{} &
			\multicolumn{1}{c|}{} &
			\multicolumn{1}{c|}{} &
			\multicolumn{1}{c|}{} &
			\multicolumn{1}{c|}{} &
			\multicolumn{1}{c|}{} &
			\multicolumn{1}{c|}{} &
			\multicolumn{1}{c|}{} &
			\multicolumn{1}{c|}{} &
			\multicolumn{1}{c|}{} &
			\multicolumn{1}{c|}{} &
			\multicolumn{1}{c|}{} &
			\multicolumn{1}{c|}{} &
			\multicolumn{1}{c|}{} &
			\\ \hline
			\makecell{$x_3 \bar x_2 \bar x_1$\\$F_3$} &
			\multicolumn{1}{c|}{} &
			\multicolumn{1}{c|}{} &
			\multicolumn{1}{c|}{} &
			\multicolumn{1}{c|}{} &
			\multicolumn{1}{c|}{} &
			\multicolumn{1}{c|}{} &
			\multicolumn{1}{c|}{} &
			\multicolumn{1}{c|}{} &
			\multicolumn{1}{c|}{} &
			\multicolumn{1}{c|}{} &
			\multicolumn{1}{c|}{} &
			\multicolumn{1}{c|}{} &
			\multicolumn{1}{c|}{} &
			\multicolumn{1}{c|}{} &
			\multicolumn{1}{c|}{} &
			\multicolumn{1}{c|}{} &
			\multicolumn{1}{c|}{} &
			\multicolumn{1}{c|}{} &
			\multicolumn{1}{c|}{} &
			\multicolumn{1}{c|}{} &
			\\ \hline
			\makecell{$x_3 \bar x_2 \bar x_1 \bar x_0$\\$F_1 F_2$} &
			\multicolumn{1}{c|}{} &
			\multicolumn{1}{c|}{} &
			\multicolumn{1}{c|}{} &
			\multicolumn{1}{c|}{} &
			\multicolumn{1}{c|}{} &
			\multicolumn{1}{c|}{} &
			\multicolumn{1}{c|}{} &
			\multicolumn{1}{c|}{} &
			\multicolumn{1}{c|}{} &
			\multicolumn{1}{c|}{} &
			\multicolumn{1}{c|}{} &
			\multicolumn{1}{c|}{} &
			\multicolumn{1}{c|}{} &
			\multicolumn{1}{c|}{} &
			\multicolumn{1}{c|}{} &
			\multicolumn{1}{c|}{} &
			\multicolumn{1}{c|}{} &
			\multicolumn{1}{c|}{} &
			\multicolumn{1}{c|}{} &
			\multicolumn{1}{c|}{} &
			\\ \hline
			\makecell{$x_3 \bar x_2 \bar x_1 x_0$\\$F_2 F_3$} &
			\multicolumn{1}{c|}{} &
			\multicolumn{1}{c|}{} &
			\multicolumn{1}{c|}{} &
			\multicolumn{1}{c|}{} &
			\multicolumn{1}{c|}{} &
			\multicolumn{1}{c|}{} &
			\multicolumn{1}{c|}{} &
			\multicolumn{1}{c|}{} &
			\multicolumn{1}{c|}{} &
			\multicolumn{1}{c|}{} &
			\multicolumn{1}{c|}{} &
			\multicolumn{1}{c|}{} &
			\multicolumn{1}{c|}{} &
			\multicolumn{1}{c|}{} &
			\multicolumn{1}{c|}{} &
			\multicolumn{1}{c|}{} &
			\multicolumn{1}{c|}{} &
			\multicolumn{1}{c|}{} &
			\multicolumn{1}{c|}{} &
			\multicolumn{1}{c|}{} &
			\\ \hline
			&
			\multicolumn{1}{c|}{} &
			\multicolumn{1}{c|}{} &
			\multicolumn{1}{c|}{} &
			\multicolumn{1}{c|}{} &
			\multicolumn{1}{c|}{} &
			\multicolumn{1}{c|}{} &
			\multicolumn{1}{c|}{} &
			\multicolumn{1}{c|}{} &
			\multicolumn{1}{c|}{} &
			\multicolumn{1}{c|}{} &
			\multicolumn{1}{c|}{} &
			\multicolumn{1}{c|}{} &
			\multicolumn{1}{c|}{} &
			\multicolumn{1}{c|}{} &
			\multicolumn{1}{c|}{} &
			\multicolumn{1}{c|}{} &
			\multicolumn{1}{c|}{} &
			\multicolumn{1}{c|}{} &
			\multicolumn{1}{c|}{} &
			\multicolumn{1}{c|}{} &
			\\ \hline
		\end{tabular}
	\end{table}
\end{landscape}
\normalsize


Найдем все простые импликанты системы логических функций \eqref{eq: FAL}, включая и функции $F_1\cdot F_2, F_1 \cdot F_3, F_2\cdot F_3, F_1 \cdot F_2 \cdot F_3$. Из диаграмм (рис.~\ref{fig:Veitch_all}) нетрудно видеть, что:
\begin{equation} \label{eq:sysFiFj}
	\left\{
	\begin{aligned}
		F_1\cdot F_2 &= x_0 \bar x_3,\\
		F_1\cdot F_3 &= x_0 \bar x_3 \vee \bar x_0 \bar x_1 \bar x_2 x_3,\\
		F_2\cdot F_3 &= x_0 \bar x_3 \vee x_0 \bar x_1 \bar x_2 x_3,\\
		F_1 \cdot F_2 \cdot F_3 &= x_0 \bar x_3.
	\end{aligned}
	\right.
\end{equation}



Теперь при помощи импликантной матрицы системы функций (табл.~\ref{tab:matrix}) определим минимальное представление системы логических функций \eqref{eq:sys}--\eqref{eq:sysFiFj}. Анализ заполненной матрицы показывает, что минимальная совокупность переключательных функций останется в виде \eqref{eq:sys}.

\subsection{Комбинационная схема, временная диаграмма}

Перейдем в базис штриха Шеффера:
\begin{equation} \label{eq:sysNand}
	\left\{
	\begin{aligned}
		F_1(x_3, x_2, x_1, x_0) &= (x_0 | \bar x_3) | (\bar x_0 | x_3),\\
		F_2(x_3, x_2, x_1, x_0) &= (x_0 | \bar x_3) |(x_0 | x_1 | x_2) | (x_0 | \bar x_1 | \bar x_2) | (x_1 | \bar x_2 | \bar x_3),\\
		F_3(x_3, x_2, x_1, x_0) &= (x_0 | \bar x_3) | (\bar x_1 | \bar x_2 | x_3).
	\end{aligned}
	\right.
\end{equation}
В соответствии с \eqref{eq:sysNand} построим логическую схему (рис.~\ref{fig:circuit}).
\begin{figure}[h!]
	%\centering\includegraphics[draft, width=15cm, height=19.5cm]{circuit}
	\centering\rectangle{15cm}{19.5cm}
	\caption{Реализация многовыходной комбинационной схемы} \label{fig:circuit}
\end{figure}

Произведем ранжирование элементов схемы (рис.~\ref{fig:circuit}):
\begin{center}
\begin{tabular}{ll}
	%\hline
	0-й ранг: & $x_3, x_2, x_1, x_0$ \\
	%\hline
	1-й ранг: & $\bar x_3, \bar x_2, \bar x_1, \bar x_0, D_3$ \\
	%\hline
	2-й ранг: & $D_1, D_2, D_4, D_5, D_6$ \\
	%\hline
	3-й ранг: & $F_1, F_2, F_3$ \\
	%\hline
\end{tabular}
\end{center}
В соответствии с рангами построим временную диаграмму (рис.~\ref{fig:timeDiagram}).

Оценим максимальные задержки переключения сигналов для каждой из функций.
\begin{itemize}[noitemsep,topsep=0pt]
	\item Функция $F_1$:
	Вычисления производились на 10-м наборе. Схема прохождения сигнала: $x_0 \rightarrow \bar x_0 \rightarrow D_2 \rightarrow F_1$. Максимальные величины задержек переключения: $t_{01} = 8~нс$, $t_{10} = 7~нс$. 
	
	\item Функция $F_2$:
	Вычисления производились на 5-м наборе. Схема прохождения сигнала: $x_0  \rightarrow D_1 \rightarrow F_2$. Максимальные величины задержек переключения: $t_{01} = 6~нс$, $t_{10} = 5~нс$.
	
	\item Функция $F_2$:
	Вычисления производились на 8-м и 9-м наборах. Схемы прохождения сигнала: $x_2 \rightarrow \bar x_2 \rightarrow D_6 \rightarrow F_3$, $x_1 \rightarrow \bar x_1 \rightarrow D_6 \rightarrow F_3$. Максимальные величины задержек переключения: $t_{01} = 8~нс$, $t_{10} = 7~нс$.
\end{itemize}

\begin{figure}
	\rectangle{17cm}{22.5cm}
	\caption{Временная диаграмма работы комбинационной схемы} \label{fig:timeDiagram}
\end{figure}

\section{Описание комбинационной схемы на VHDL}

Описать комбинационную схему на языке VHDL можно различными способами. Воспользуемся одним из предложенных в лабораторном практикуме, а именно "--- с использованием \emph{параллельного сигнального оператора присваивания}, при этом функции будем описывать в виде \emph{булевых уравнений} (листинг~\ref{list:prog}). Удобнее всего при этом воспользоваться ранее полученным представлением функций в форме МДНФ \eqref{eq:sys}.

\begin{lstlisting}[language=VHDL, caption={Описание комбинационной схемы на языке VHDL}\label{list:prog}]
	entity LAB2_KS is
		port (
			 x3, x2, x1, x0: in BIT;
			 F1, F1, F3: out BIT
			 );
	end LAB2_KS;
	
	architecture LAB2_KS_arch of LAB2_KS is
	begin
		F1 <= (not x3 and x0) or (x3 and not x0);
		F2 <= (not x3 and x0)
			  or (x2 and x1 and x0)
			  or (not x2 and not x1 and x0)
			  or (not x3 and not x2 and x1);
		F3 <= (not x3 and x0)
			  or (x3 and not x2 and not x1);
	end LAB2_KS_arch;
\end{lstlisting}

\end{document}